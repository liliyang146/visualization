%% LyX 2.1.3 created this file.  For more info, see http://www.lyx.org/.
%% Do not edit unless you really know what you are doing.
\documentclass[a4paper]{article}
\usepackage{ucs}
\usepackage[utf8x]{inputenc}
\usepackage[sc]{mathpazo}
\usepackage[T1]{fontenc}
\usepackage{geometry}
\usepackage{wasysym}
\usepackage{longtable}
\usepackage{newtxtext}


% \usepackage{fontspec} % loaded by polyglossia, but included here for transparency 
% \usepackage{polyglossia}
% \setmainlanguage{russian} 
% \setotherlanguage{english}


\bibliographystyle{alpha}


\newenvironment{uzdevums}[1][\unskip]{%
\vspace{3mm}
\noindent
\textbf{#1 uzdevums:}
\noindent}
{}


\geometry{verbose,tmargin=2.5cm,bmargin=2.5cm,lmargin=2.5cm,rmargin=2.5cm}
\setcounter{secnumdepth}{2}
\setcounter{tocdepth}{2}
\usepackage{url}
\usepackage[unicode=true,pdfusetitle,
bookmarks=true,bookmarksnumbered=true,bookmarksopen=true,bookmarksopenlevel=2,
breaklinks=false,pdfborder={0 0 1},backref=false,colorlinks=false]
{hyperref}
\renewcommand{\abstractname}{Anotācija}
\renewcommand{\figurename}{Attēls}
\renewcommand{\tablename}{Tabula}
\renewcommand{\refname}{Atsauces}
\setlength{\parindent}{0cm}


\hypersetup{
pdfstartview={XYZ null null 1}}
\usepackage{Sweave}
\begin{document}
\input{dzejas-analize-ar-statistiku-concordance}
\begin{Schunk}
\begin{Sinput}
> library(knitr)
> # set global chunk options
> opts_chunk$set(fig.path='figure/minimal-', fig.align='center', fig.show='hold')
> options(formatR.arrow=TRUE,width=90)
\end{Sinput}
\end{Schunk}


\title{Dzejas analīze ar statistiku}

\author{}
\date{}

\maketitle

\begin{abstract}
Šajā dokumentā salīdzinātas O.Vācieša un Raiņa dzejoļu
statistiskās īpašības, kuras viegli izmērīt ar datoru. 
Analizēti divi dzejoļu krājumi --- O.Vācieša {\em Dzegužlaiks} 
(ap 100 tūkstoši rakst\-zīmju) un 
Raiņa {\em Tālas noskaņas zilā vakarā} (ap 60 tūkstoši rakst\-zīmju). 
Turpmāk apzīmēsim tos attiecīgi ar {\em Dz-L} un {\em TNZV}. 

Abos krājumos skaitīts burtu biežums, meklēti dzejoļi vai to fragmenti, 
kuros ir bieži sastopams kāds burts/skaņa, salīdzināts vārdformu biežums 
kā arī veikts salīdzinājums krāsas izsakošiem vārdiem - to relatīvajam 
biežumam un morfolo\v{g}iskajām īpašībām. 
\end{abstract}

